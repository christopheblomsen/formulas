% Standard stuff
	\documentclass[a4paper,10pt,english]{article}
	\usepackage[utf8]{inputenc}
	\usepackage[norsk]{babel}
	\usepackage{amsmath,graphicx,varioref,verbatim,amsfonts,geometry,enumerate,commath}
	% colors in text
	\usepackage[usenames,dvipsnames,svgnames,table]{xcolor}
	% Hyper refs
	\usepackage[colorlinks]{hyperref}
	%inkspace
	\usepackage{import}
	\usepackage{xifthen}
	\usepackage{pdfpages}
	\usepackage{transparent}

%Colour scheme for hyperlinks
	\hypersetup{%
		colorlinks,
		citecolor=Blue,
		linkcolor=Blue,
		urlcolor=Blue}

% Document formatting
	\setlength{\parindent}{0mm}
	\setlength{\parskip}{1.5mm}

%Color scheme for listings
	\usepackage{textcomp}
	\definecolor{listinggray}{gray}{0.9}
	\definecolor{lbcolor}{rgb}{0.9,0.9,0.9}

%Listings configuration
	\usepackage{listings}
	\lstset{
		backgroundcolor=\color{lbcolor},
		tabsize=4,
		rulecolor=,
		language=python,
		basicstyle=\scriptsize,
		upquote=true,
		aboveskip={1.5\baselineskip},
		columns=fixed,
		numbers=left,
		showstringspaces=false,
		extendedchars=true,
		breaklines=true,
		prebreak = \raisebox{0ex}[0ex][0ex]{\ensuremath{\hookleftarrow}},
		frame=single,
		showtabs=false,
		showspaces=false,
		showstringspaces=false,
		identifierstyle=\ttfamily,
		keywordstyle=\color[rgb]{0,0,1},
		commentstyle=\color[rgb]{0.133,0.545,0.133},
		stringstyle=\color[rgb]{0.627,0.126,0.941}
	}

%new commands
	\newcommand{\dd}[1]{\mathrm{d}#1}
	\newcommand{\uvec}[1]{\boldsymbol{\hat{\textbf{#1}}}}

	\newcommand{\incfig}[2][1]{%
		\def\svgwidth{#1\columnwidth}
		\import{./figures/}{#2.pdf_tex}
	}
	\pdfsuppresswarningpagegroup = 1

%opening
	\title{Formler fys1120}
	\author{
		Christophe Blomsen\\
		\texttt{\href{mailto:chriskbl@student.matnat.uio.no}{chriskbl@student.matnat.uio.no}}
		}

\begin{document}
%Titlepage
	\begin{titlepage}
	\maketitle
	\tableofcontents
	\end{titlepage}


\section{Elektrisk felt}

\begin{align*}
	\vec{F} = \frac{Qq}{4 \pi \epsilon_0}	\frac{ \uvec{R}}{R^2} = \frac{ Qq}{4 \pi \epsilon_0} \frac{ \vec{R}}{R^3}
\end{align*}

\[
	\vec{E}= \frac{ \vec{F}}{q} = \frac{Q}{4 \pi \epsilon_0} \frac{\uvec{R}}{R^3}
\] 

\[
	\dd{ \vec{E}} = \frac{ \dd{ q} }{4 \pi \epsilon_0} \frac{ \uvec{R}}{R^2}
\] 

\[
\rho_l = \frac{ Q}{L}
\] 

\[
	V(\vec{r}) = \int \limits_{\vec{r}}^{ref} \vec{E} \cdot \dd{ \vec{l}} 
\] 

\[
	\oint \limits_C \vec{E} \cdot \vec{l} = 0 \Leftrightarrow \iint \limits_S \nabla \times E \cdot \dd{ S} =0
\] 
\section{Elekrisk potensial}

\[
	E = - \nabla V
\] 
\[
	E = \int \limits_V \rho \frac{ \dd{ v} }{4 \pi \epsilon_0} \frac{ \uvec{R}}{R^2}
\] 
\[
	V = \int \frac{\rho \dd{ V} }{4 \pi \epsilon_0 R}
\] 

\[
	V(\vec{r}) = \frac{ \vec{p} \cdot \uvec{r}}{4 \pi \epsilon_0 r^2}
\] 
\section{Gauss lov og dielektriske materialer}
\[
\Delta v = \Delta s d \cos( \theta) 
\] 
Antall dipoler i $\Delta v$
 \[
	 Nv \Delta v = Nv \Delta s \vec{d} \cdot \uvec{n}
\] 
Hvor mye ladning
\begin{align*}
	\Delta Q_b &= -Q \cdot Nv \Delta s \vec{d} \cdot \uvec{n}\\
		   &= -Q \cdot Nv \vec{d} \cdot \Delta \vec{s} \\
		   &= -Nv \left( Q \vec{d} \right) \cdot \Delta \vec{s}\\
		   &= - Nv \vec{p} \cdot \Delta \vec{s}\\
		   &= - \frac{ N \vec{p}}{v} \cdot \Delta \vec{s} \\
		   &= - \vec{p} \cdot \Delta \vec{s}
\end{align*} 
\[
	Q_b = - \oint \limits_S \vec{p} \cdot \dd{ \vec{s}} 
\] 

Bundet volum-ladningstethet
\begin{align*}
	Q_b = \int \limits_v \rho_{v,b} \dd{ v} &= - \oint \limits_s \vec{p} \cdot \dd{ \vec{s}} \\
						&= - \oint \limits_S \vec{p} \cdot \dd{ \vec{s}}\\
						&= - \int\limits_v \nabla \cdot \vec{p} \dd{ v} 
\end{align*}
\[
	\boxed{\rho_{v,b} = - \nabla \cdot \vec{p}}
\] 
\end{document}
