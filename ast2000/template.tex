% Standard stuff
	\documentclass[a4paper,10pt,english]{article}
	\usepackage[utf8]{inputenc}
	\usepackage[norsk]{babel}
	\usepackage{amsmath,graphicx,varioref,verbatim,amsfonts,geometry,enumerate,commath}
	% colors in text
	\usepackage[usenames,dvipsnames,svgnames,table]{xcolor}
	% Hyper refs
	\usepackage[colorlinks]{hyperref}
	%inkspace
	\usepackage{import}
	\usepackage{xifthen}
	\usepackage{pdfpages}
	\usepackage{transparent}

%Colour scheme for hyperlinks
	\hypersetup{%
		colorlinks,
		citecolor=Blue,
		linkcolor=Blue,
		urlcolor=Blue}

% Document formatting
	\setlength{\parindent}{0mm}
	\setlength{\parskip}{1.5mm}

%Color scheme for listings
	\usepackage{textcomp}
	\definecolor{listinggray}{gray}{0.9}
	\definecolor{lbcolor}{rgb}{0.9,0.9,0.9}

%Listings configuration
	\usepackage{listings}
	\lstset{
		backgroundcolor=\color{lbcolor},
		tabsize=4,
		rulecolor=,
		language=python,
		basicstyle=\scriptsize,
		upquote=true,
		aboveskip={1.5\baselineskip},
		columns=fixed,
		numbers=left,
		showstringspaces=false,
		extendedchars=true,
		breaklines=true,
		prebreak = \raisebox{0ex}[0ex][0ex]{\ensuremath{\hookleftarrow}},
		frame=single,
		showtabs=false,
		showspaces=false,
		showstringspaces=false,
		identifierstyle=\ttfamily,
		keywordstyle=\color[rgb]{0,0,1},
		commentstyle=\color[rgb]{0.133,0.545,0.133},
		stringstyle=\color[rgb]{0.627,0.126,0.941}
	}

%new commands
	\renewcommand{\lstlistingname}{Kode}
	\renewcommand{\lstlistlistingname}{\lstlistingname}
	\newcommand{\dd}[1]{\mathrm{d}#1}
	\def\doubleunderline#1{\underline{\underline{#1}}}
	\renewcommand{\thesection}{Oppgave \arabic{section}}
	\newcommand{\uvec}[1]{\boldsymbol{\hat{\textbf{#1}}}}

	\newcommand{\incfig}[2][1]{%
		\def\svgwidth{#1\columnwidth}
		\import{./figures/}{#2.pdf_tex}
	}
	\pdfsuppresswarningpagegroup = 1

%opening
	\title{}
	\author{
		Christophe Blomsen\\
		\texttt{\href{mailto:chriskbl@student.matnat.uio.no}{chriskbl@student.matnat.uio.no}}
		}

\begin{document}
%Titlepage
	\begin{titlepage}
	\maketitle
	\tableofcontents
	\listoffigures
	\lstlistoflistings
	\end{titlepage}

%Oppgave 1
	\section\\ 
	Regn ut flateintegralet\[
	\iint\limits_T x \text{ } \dd{S}
	\] 
	der $T$ er delen av sfæren $x^2+y^2+z^2=a^2$ som ligger i første oktant $x\leq 0, y\leq 0, z\leq 0$\\
	\newline
	Siden det er en sfære vi skal regne på så konverterer jeg det over til kulekoordinater.\[
	x=\rho \cos(\theta) \sin(\phi) \wedge y = \rho \sin(\theta)\sin(\phi) \wedge z = \rho \cos(\phi)  
	\] 
	Kommer så til å bruke sammenhengen \[
		\iint\limits_T f \text{ } \dd{S} = \iint\limits_A f \left( \mathbf{r} (u, v) \right) \abs{\left( \frac{\partial \mathbf{r}}{\partial u} \times \frac{\partial \mathbf{r}}{\partial v } \right) (u,v)  } \dd{u}\dd{v}
	\]  
	Hvor $u = \phi$ og $v = \theta$\\
	Siden formelen for en kule er $x^2+y^2+z^2=\rho^2$ så medfører det at $\rho=a$. Videre siden vi kun ser på første oktant så må $\theta \in [0, \frac{\pi}{2}]$ og $\phi \in [0, \frac{\pi}{2}]$. Så parametriseringen blir \[
		\mathbf{r}(\theta,\phi) = (a \cos(\theta) \sin(\phi), a \sin(\theta) \sin(\phi), a \cos(\phi))
	\] 
	Videre så blir de partiell deriverte \[
		\frac{\partial \mathbf{r}}{\partial \phi} = (a \cos(\theta) \cos(\phi), a \sin(\theta) \cos(\phi)  ,-a \sin(\theta) \cos(\phi) ) \wedge \frac{\partial \mathbf{r}}{\partial \theta}=(-a \sin(\theta) \sin(\phi), a \cos(\theta) \sin(\phi), 0)
	\] 
	Dette fører da til
	\begin{align*}
		&\frac{\partial \mathbf{r}}{\partial \phi} \times \frac{\partial \mathbf{r}}{\partial \theta} = (a^2 \cos(\theta) \sin^2(\phi)) \uvec{i} + (a^2 \sin^2(\phi)\sin(\theta)) \uvec{j} + (a^2 \cos^2(\theta)\cos(\phi) \sin(\phi) + a^2 \sin^2(\theta) \sin(\phi) \cos(\phi)) \uvec{k} \\ 
		&\frac{\partial \mathbf{r}}{\partial \phi} \times \frac{\partial \mathbf{r}}{\partial \theta} = (a^2 \cos(\theta) \sin^2(\phi)) \uvec{i} + (a^2 \sin^2(\phi)\sin(\theta)) \uvec{j} + (a^2 \cos(\phi) \sin(\phi) ) \uvec{k}
	\end{align*}
	Som betyr at
	\begin{align*}
		&\abs{\frac{\partial \mathbf{r}}{\partial \phi} \times \frac{\partial \mathbf{r}}{\partial \theta} }=\sqrt{\left(a^2 \cos(\theta) \sin^2(\phi)\right)^2 + \left (a^2 \sin^2(\phi)\sin(\theta) \right)^2 + \left (a^2 \cos(\phi) \sin(\phi)\right )^2}\\
		&\abs{\frac{\partial \mathbf{r}}{\partial \phi} \times \frac{\partial \mathbf{r}}{\partial \theta} }=\sqrt{a^4 \cos^2(\theta)\sin^4(\phi) + a^4 \sin^4(\phi)\sin^2(\theta) + a^4 \cos^2(\phi) \sin^2(\phi)}\\
		&\abs{\frac{\partial \mathbf{r}}{\partial \phi} \times \frac{\partial \mathbf{r}}{\partial \theta} }=\sqrt{a^4 \sin^2(\phi) \left (\cos^2(\theta)\sin^2(\phi) +\sin^2(\phi)\sin^2(\theta)+ \cos^2(\phi)\right )}\\
		&\abs{\frac{\partial \mathbf{r}}{\partial \phi} \times \frac{\partial \mathbf{r}}{\partial \theta} }=a^2 \sin(\phi) \sqrt{\sin^2(\phi) \left (\cos^2(\theta) + \sin^2(\theta) \right ) + \cos^2(\phi)}\\
		&\abs{\frac{\partial \mathbf{r}}{\partial \phi} \times \frac{\partial \mathbf{r}}{\partial \theta} }=a^2 \sin(\phi) \sqrt{\sin^2(\phi) + \cos^2(\phi)}\\
		&\abs{\frac{\partial \mathbf{r}}{\partial \phi} \times \frac{\partial \mathbf{r}}{\partial \theta} }=a^2 \sin(\phi)
	\end{align*}
	Kan så se på dobbeltintegralet
	\begin{align*}
		&\int\limits_0^{\frac{\pi}{2}} \int\limits_0^{\frac{\pi}{2}}a \cos(\theta) \sin(\phi) \cdot a^2 \sin(\phi) \dd{\phi}\dd{\theta} =\\
		&\int\limits_0^{\frac{\pi}{2}} \int_0^{\frac{\pi}{2}} a^3 \cos(\theta) \sin^2(\phi) \dd{\phi}\dd{\theta}=\\
		&a^3\int\limits_0^{\frac{\pi}{2}} \cos(\theta) \dd{\theta} \int_0^{\frac{\pi}{2}}\frac{1-\cos(2 \phi)}{2}\dd{\phi}=\\
		&\frac{1}{2}a^3\int\limits_0^{\frac{\pi}{2}}\cos(\theta) \dd{\theta} \left [ \int_0^{\frac{\pi}{2}}1 \dd{\phi}-\int_0^{\frac{\pi}{2}}\cos(2 \phi)\dd{\phi} \right ]=\\
		&\boxed{u=2 \phi \Rightarrow \frac{\dd{u}}{\dd{\phi}}=2 \Leftrightarrow \dd{\phi}= \frac{\dd{u}}{2}}\\
		&\frac{1}{2}a^3\int\limits_0^{\frac{\pi}{2}}\cos(\theta) \dd{\theta} \left [ \int_0^{\frac{\pi}{2}}1 \dd{\phi}- \int_0^\pi \cos(u) \dd{u} \right ]=\\
		&\frac{1}{2}a^3\int_0^{\frac{\pi}{2}}\cos(\theta) \dd{\theta} \left [\frac{\pi}{2} - \left. \sin(u) \right |_0^\pi \right ] =\\
		&\frac{1}{2}a^3\int_0^{\frac{\pi}{2}}\cos(\theta)\frac{\pi}{2}\dd{\theta}=\\
		&\frac{\pi}{4}a^3\int_0^{\frac{\pi}{2}}\cos(\theta)\dd{\theta}=\\
		&\frac{\pi}{4}a^3 \left [ \left. \sin(\theta) \right |_0^{\frac{\pi}{2}} \right ]=\\ 
		&\frac{\pi}{4}a^3 \left [1 -0 \right ]= \frac{\pi}{4}a^3
	\end{align*}
	\newpage
%Oppgave 2
	\section\\ 
	En dobbelt deriverbar funksjon $f:\mathbb{R}^2 \rightarrow \mathbb{R}$ kalles \emph{harmonisk} dersom den tilfredstiller \emph{Laplace-likningen}
	\[
	\frac{\partial ^2 f}{\partial x^2}+\frac{\partial ^2 f}{\partial y^2}=0
	\] 
	\begin{enumerate}[a)]
		\item Vis at funksjonene $x^4 - 6x^2y^2 +y^4$ og $\mathrm{e}^x\sin(y)$ er harmoniske.\\
			\newline
			Starter med $f(x,y)=x^4 - 6x^2y^2 +y^4$ 
			\begin{align*}
				&\frac{\partial f}{\partial x}=4x^3-12xy^2
				&\wedge &
				&\frac{\partial f}{\partial y}=-12x^2y +4y^3\\
				&\frac{\partial ^2 f}{\partial x^2}=12x^2 -12y^2
				&\wedge &
				&\frac{\partial ^2 f}{\partial y^2}=-12x^2+12y^2
			\end{align*}
			Dette medfører til:
			\begin{align*}
				&\frac{\partial ^2 f}{\partial x^2}+\frac{\partial ^2 f}{\partial y^2}=0\\
				&12x^2 -12y^2+-12x^2+12y^2=0
			\end{align*}
			Som betyr at $f(x,y)$ er harmonisk.\\
			\newline
			Ser så på $g(x,y)= \mathrm{e}^x\sin(y)$
			\begin{align*}
				&\frac{\partial g}{\partial x}=e^x \sin(y) 
				& \wedge &
				&\frac{\partial g}{\partial y} = \mathrm{e}^x \cos(y)\\
				&\frac{\partial ^2 g}{\partial x^2} = \mathrm{e}^x \sin(y)
				& \wedge &
				&\frac{\partial ^2 g}{\partial y^2} = - \mathrm{e}^x \sin(y)
			\end{align*}
			Dette medfører til:
			\begin{align*}
				&\frac{\partial ^2 g}{\partial x^2}+\frac{\partial ^2 g}{\partial y^2}=0\\
				&e^x \sin(y)-e^x \sin(y)=0
			\end{align*}
			Som betyr at $g(x,y)$ også er harmonisk.
		\item Vis at dersom $f$ er harmonisk, så er \[
			\int_C \left ( \frac{\partial f}{\partial y} \dd{x} - \frac{\partial f}{\partial x} \dd{y} \right ) = 0
		\] 
			for all enkle, lukkede, glatte kurver $C$ i planet.\\
			\newline
			Tenker å bruke Greens Teorem som sier \[
				\int_C P \dd{x} + Q \dd{y} = \iint_R \left (\frac{\partial Q}{\partial x} - \frac{\partial P}{\partial y} \right ) \dd{x}\dd{y} 
			\] 
			I mitt tilfelle er \[
			P = \frac{\partial f}{\partial y} \wedge Q = -\frac{\partial f}{\partial x} 
			\] 
			Så ved innsetning i Greens Teorem
			\begin{align*}
				& \int\limits_C P \dd{x} + Q \dd{y} = \\
				& \int\limits_C \left (\frac{\partial f}{\partial y} \dd{x} - \frac{\partial f}{\partial x} \dd{y} \right) = \\
				& \iint\limits_R \left( \frac{\partial }{\partial x} \left( - \frac{\partial f}{\partial x} \right) - \frac{\partial }{\partial y} \left( \frac{\partial f}{\partial y}  \right) \right) \dd{x}\dd{y} =\\ 
				& \iint\limits_R \left( - \frac{\partial^2 f}{\partial x^2} - \frac{\partial^2 f}{\partial y^2} \right) \dd{x}\dd{y} = \\
				& \iint\limits_R - \left( \frac{\partial^2 f}{\partial x^2} + \frac{\partial^2 f}{\partial y^2} \right) \dd{x}\dd{y} = \\
				& \text{Fra a) har vi at } \frac{\partial^2 f}{\partial x^2} + \frac{\partial^2 f}{\partial y^2} = 0 \text{ når $f$ er harmonisk}\\
				& \iint\limits_R 0 \text{ } \dd{x}\dd{y} = 0
			\end{align*}
			Linjeintegralet må være null
	\end{enumerate}
	\newpage
%Oppgave 3

			\begin{figure}[h!]
			    \centering
			    \label{fig:delmengde-a}
			    \incfig{delmengde-a}
			    \caption{Delmengde $A$}
			\end{figure}\\
	\section \\
	La $A = [-1, 1] \times [-1, 1]$ og la $\mathbf{F}:A \rightarrow \mathbb{R}^2$ være gitt ved \[
		\mathbf{F}(x,y) = \frac{1}{5}\left( xy + y^2 -1, x^3 - y^2 +3 \right)
	\] 
	\begin{enumerate}[a)]
		\item Vis at \textbf{F} definerer en funksjon $\mathbf{F}: A \rightarrow A$\\
			\newline
			Alle verdiene til $A$ er inne i kvadratet med hjørner $(-1, -1), (1, -1), (1, 1), (-1, 1)$. Så hvis disse $(x,y)$-verdiene er inne i $A$ etter å ha gått gjennom $\mathbf{F}$ så må alle verdiene til $A$ også være det.
			\begin{align}
				\mathbf{F}(-1, -1) &= \frac{1}{5} \left( \left(-1\right) (-1) + (1)^2 -1, (-1)^3 - (-1)^2 + 3\right) = \frac{1}{5} \left( 1, 1 \right) \\
				\mathbf{F}(1, -1) &= \frac{1}{5} \left( 1 \cdot (-1)  + (-1)^2 - 1, 1 - \left( -1 \right)^2 + 3 \right) = \frac{1}{5} \left( -1, 3 \right) \\
				\mathbf{F}(1, 1) &= \frac{1}{5} \left( 1 \cdot 1 + 1^2 -1, 1^3 - 1^2 + 3 \right) = \frac{1}{5} \left( 1, 3 \right) \\
				\mathbf{F}(-1, 1) &= \frac{1}{5} \left( (-1) \cdot 1 + 1^2 -1, (-1)^3 -1^2 +3 \right) = \frac{1}{5}(-1, 1)
			\end{align}
			Ser da at alle ligningene over gir verdier som er inne i $A$. Som betyr at $\mathbf{F}$ definerer en funksjon $\mathbf{F}: A \rightarrow A$. Dette kan også gjøres med trekantulikheten. Altså at 
			\begin{align*}
				&\abs{F_1} = \abs{\frac{1}{5} \left( xy + y^2 - 1 \right)} \\
				&\abs{F_1} \leq \frac{1}{5} \left(  \abs{xy} + \abs{y^2} + \abs{-1} \right)  \\
				&\frac{1}{5} \left(\abs{xy} + \abs{y^2} + \abs{-1}\right) \leq \frac{1}{5} \left( \abs{x} \abs{y} + \abs{y} \abs{y} +  \abs{-1} \right)\\
				&\text{Siden høyeste verdien $A$ har er $1$ }\\
				&\abs{F_1} \leq \frac{1}{5} \left( 1 + 1 +1 \right) = \frac{3}{5} \leq 1
			\end{align*}
			Kan gjøre tilsvarende på $F_2$
			\begin{align*}
				&\abs{F_2} = \abs{\frac{1}{5} \left( x^3 - y^2 + 3 \right)} \\
				&\abs{F_2} \leq \frac{1}{5} \left(  \abs{x^3} + \abs{-y^2} + \abs{3} \right)  \\
				&\frac{1}{5} \left(  \abs{x^3} + \abs{-y^2} + \abs{3} \right) \leq \frac{1}{5} \left( \abs{x} \abs{x} \abs{x} + \abs{y} \abs{y} + \abs{3} \right)\\
				&\text{Siden høyeste verdien $A$ har er $1$ }\\
				&\abs{F_2} \leq \frac{1}{5} \left( 1 + 1 + 3\right) = \frac{5}{5} \leq 1
			\end{align*}
		\item Vis at $\mathbf{F}: A \rightarrow A$ definerer en kontraksjon \\
			\newline
			Ser at det kan velges 2 tilfeldige punkter inne i delmengden $A$ for å så trekke en rett strek mellom disse punktene uten at streken blir brutt. Så delmengden $A$ er konveks.\\
			Siden da $A$ er en ikke-tom, lukket, konveks delmengde av $\mathbb{R}^2$ og at $\mathbf{F}: A \rightarrow A$ er en avbildning som er deriverbar i $A$ så definerer $\mathbf{F}$ en kontraksjon. Dette kan også sees med at man bruker setning 5.5.8 i FLVA som sier \[
				\sqrt{\abs{\nabla \mathbf{F}_1 (\mathbf{c}_1)}^2 + \abs{\nabla \mathbf{F}_2 (\mathbf{c}_2)}^2} \leq C
			\]
			Hvor $C < 1$. Starter da med å finne Jacobi matrisen til $\mathbf{F}$
			\begin{align*}
				&\mathbf{F}(\mathbf{x}) = \frac{1}{5}
				\begin{pmatrix}
					\frac{\partial F_1}{\partial x} & \frac{\partial F_1}{\partial y} \\
					\frac{\partial F_2}{\partial x} & \frac{\partial F_2}{\partial y} 
				\end{pmatrix}\\
				&\mathbf{F}(\mathbf{x}) = \frac{1}{5}
				\begin{pmatrix}
					y & x + 2y \\
					3x^2 & -2y
				\end{pmatrix}
			\end{align*}
			Kan så finne gradientene
			\begin{align*}
				\nabla \mathbf{F}_1 (x, y) = \frac{1}{5} \left( y \uvec{i} + (x + 2y) \uvec{j} \right) \wedge \nabla \mathbf{F}_2 (x, y) = \frac{1}{5} \left( 3x^2 \uvec{i} - 2y \uvec{j} \right) 
			\end{align*}
			Siden den maksimale verdien som  $A$ har er $(1, 1)$, så settes det inn mens setning 5.5.8 blir brukt
			\begin{align*}
				&\sqrt{\abs{\nabla \mathbf{F}_1 (1,1)}^2 + \abs{\nabla \mathbf{F}_2 (1, 1)}^2} =\\
				&\sqrt{\left( \frac{1}{5} \right)^2 + \left( \frac{3}{5} \right)^2 + \left( \frac{3}{5} \right)^2 \left( \frac{2}{5} \right)^2} =\\
				&\sqrt{\frac{1+9+9+4}{25}}=\\
				&\sqrt{\frac{23}{25}} = \frac{\sqrt{23}}{25} \leq C  < 1
			\end{align*}
			Da definer \textbf{F} en kontraksjon siden.
		\item Vis at følgende likningsystem har en unik løsning for $-1 \leq x, y \leq 1$:
			\begin{align*}
				5x &= xy + y^2 -1\\
				5y &= x^3 -y^2 +3
			\end{align*}
			Ser at hvis vi deler begge ligningene på $5$ så får vi 
			\begin{align*}
				x &= \frac{1}{5} \left(xy + y^2 - 1 \right)\\
				y &= \frac{1}{5} \left( x^3 - y^2 + 3 \right) 
			\end{align*} 
			Som er de $x$ og $y$ verdiene vi har blitt bekjent med tidligere. Dette tilsier da at \[
			\mathbf{x} = \frac{1}{5} 
			\begin{pmatrix}
				xy + y^2 - 1\\
				x^3 - y^2 + 3
			\end{pmatrix}
			\] 
			Vi veit også at vår kontraksjonsfaktor $C$ ligger i intervallet $\left[ \frac{\sqrt{23}}{25}, 1 \right)$. Videre veit vi også at $A$ er en ikke-tom, lukket delmengde av $\mathbb{R}^2$ og at $\mathbf{F}: A \rightarrow A$ er en kontraksjon av $A$ med kontraksjonsfaktor $C$. Etter Banachs fikspunktsteorem så har da \textbf{F} nøyaktig ett fikspunkt $\mathbf{x}$ i $A$. Så da har likningsystemet nøyaktig en unik løsning siden det kun er ett fikspunkt.
		\item Lag et MATLAB/python script som regner ut en approksimasjon til løsningen i c) ved hjelp av en iterasjon $\mathbf{x}_{n+1} = \mathbf{F}(\mathbf{x}_n)$. Programmet skal ta startpunkt $\mathbf{x}_0$ og antall iterasjoner som input. Legg ved et plott av følgene du får med startpunk i $(0,0)$ og $(-1,1)$.
	\end{enumerate}
	\begin{lstlisting}[caption=Kode til oppgave 3 d]
import numpy as np
import matplotlib.pyplot as plt

n = 50

timestep = np.arange(0, n + 1, 1)

first_array = np.zeros((2, n + 1))


def newtons_method(x0, y0, n):
    xn = np.zeros(n + 1)
    yn = np.zeros(n + 1)
    xn[0] = x0
    yn[0] = y0
    for i in range(n):
        xn[i + 1] = 1/5 * (xn[i]*yn[i] + yn[i]**2 - 1)
        yn[i + 1] = 1/5 * (xn[i]**3 - yn[i]**2 + 3)

    return xn, yn


x01, y01 = 0, 0
X1, Y1 = newtons_method(x01, y01, n)

x02, y02 = -1, 1
X2, Y2 = newtons_method(x02, y02, n)

print(f"x blir {X1[-1]}, og y blir {Y1[-1]}")
print(f"x blir {X2[-1]}, og y blir {Y2[-1]}")
fig, axs = plt.subplots(1, 2, sharex=True, sharey=True,
			gridspec_kw={"wspace": 0})

fig.suptitle("Oppgave 3 d)")
axs[0].plot(timestep, X1, timestep, Y1)
axs[0].set_title("$\mathbf{x}_0 = (0,0)$")

axs[1].plot(timestep, X2, timestep, Y2)
axs[1].set_title("$\mathbf{x}_0 = (-1, 1)$")

for ax in axs.flat:
    ax.set(xlabel="Iterasjoner", ylabel="Verdier")
    ax.label_outer()
    ax.legend(["x", "y"])


plt.savefig("assignment_3d.png")
plt.show()
	\end{lstlisting}
	\newpage
	Som produserer følgende plot\\
	\begin{figure}[h!]
		\centering
		\caption{Plot til oppgave 3 d)}
		\label{fig:3d}
		\includegraphics{assignment_3d.png}
	\end{figure}\\
	Finner så fra terminal utrskriften at $x \rightarrow -0.15868434412055976$ og  $y \rightarrow 0.5407242909860617$ for begge startverdiene
\end{document}
